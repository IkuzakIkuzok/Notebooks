
%% (c) 2018 Kazuki KOHZUKI

\documentclass{../../../notebook}

\begin{document}

\day{2018/04/09}

\section*{物理化学の復習}

\subsection{熱力学}

エネルギーの出入りを伴う系内の進行\\\\

基本: バルクな物体におけるエネルギーの移動と変換\\\\

\par

$\left\{\begin{array}{lll}
\mathrm{内部エネルギー} & : & U\ [J]\\
\mathrm{エントロピー} & : & S\ [J\cdot K^{-1}]
\end{array}\right.$

\paragraph{熱力学第一法則}

孤立系(外部の影響がない)においてエネルギーは一定\\\\

試料の温度が$\Delta T$上昇すると内部エネルギー変化$\Delta U$は

\begin{eqnarray*}
  \Delta U &=& C\Delta T \\
  C &=& \frac{\Delta U}{\Delta T} \cdots \mathrm{熱容量}
\end{eqnarray*}

\subparagraph{圧力一定の系}

\begin{eqnarray*}
  H &=& U + PV \\
  \Delta H &=& \Delta U + P\Delta V
\end{eqnarray*}

\hfill エンタルピー変化 \hfill $P$が一定のとき,系から熱として移動するエネルギー\\

エネルギーが多くの運動モードに分散されることで拡散が発生\\

\paragraph{熱力学第二法則}

孤立系の内部における自発的変化 (無秩序化)\\\\

$\mathrm{秩序化} \rightleftharpoons \mathrm{無秩序化}$\\
\phantom{$\mathrm{秩序化} \rightleftharpoons $}どちらが優先して作用しているのか\\
\phantom{$\mathrm{秩序化} \rightleftharpoons $}平衡状態の偏り\\
\phantom{$\mathrm{秩序化} \rightleftharpoons $}$\Delta H$,$T\Delta S$の関係

\begin{eqnarray*}
  \Delta G &=& \Delta H - T\Delta S\\
  &=& -RT \ln K
\end{eqnarray*}

\subsection{集合体と分子}

\[
  \begin{array}{c}\mathrm{集合体}\\\mathrm{固体}\end{array} \longrightarrow \mathrm{液体} \longrightarrow \mathrm{気体} \xrightarrow{\mathrm{低濃度}}
    \begin{array}{c}\mathrm{1個の単位}\\\mathrm{(完全孤立)}\end{array}
\]

$\begin{array}{lll}
\mathrm{1個の単位} & \longrightarrow & \mathrm{化学式で表している状態}\Longrightarrow\mathrm{挙動は計算でシミュレートできる。}\\
\mathrm{日常の現象} & \longrightarrow & \mathrm{分子間の相互作用を平均化して観察または測定している}
\end{array}$

\[\begin{array}{ccc}
\mathrm{量子化学} & \cdots & \mathrm{1個の単位 (単純化)} \\
\downarrow & & \downarrow \\
\mathrm{実在} & & \mathrm{集合・相互作用} \\
& & \downarrow \\
& & \mathrm{平均化された物性値}
\end{array}\]

\underline{量子化}される\\
\hspace{1zw}{\it quantization}

\hspace{10zw}量子の世界の挙動\\
\hspace{11zw}1個の量子の持つ\underline{エネルギー}\\
\hspace{11zw}\phantom{1個の量子の持つ}不連続=離散的

\[
  \mathrm{分子1個} \longrightarrow \mathrm{分子1個} \longrightarrow
  \begin{array}{l}
    \mathrm{原子核1個と個々の電子}\\
    (\mathrm{量子} \to \mathrm{素粒子})
  \end{array}
\]

分子1個のもつエネルギー\\
\hspace{5zw}(ボトムアップ)

\end{document}
